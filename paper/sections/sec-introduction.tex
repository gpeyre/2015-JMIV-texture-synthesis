\section{Introduction}
\label{sec:intro}


\subsection{Texture Synthesis}

Texture synthesis is the process of generating new texture images from a given image sample.
Being able to perform such a synthesis is of practical interest both for computer graphics, where it is needed to give a realistic aspect to 3D models, or for image restoration, where it may help the reconstruction of missing parts through inpainting.

One of the most difficult aspect of texture synthesis is the ability to deal with the wide variety of texture images.
In particular, it is relatively difficult to synthesize textures at all observation scales using a given model.
Before being more specific in the next section, we can roughly split texture synthesis methods in two groups.
On the one hand, non-parametric patch-based methods, initiated by the seminal works from Efros and Leung~\cite{efros1999texture} and Wei and Levoy~\cite{wei2000fast}, have proven their efficiency to synthesize highly structured textures, such as brick walls or pebbles seen at a close distance.
On the other hand, methods relying on statistical constraints, initiated by the work of Heeger and Bergen~\cite{Heeger1995}, provide efficient methods for the synthesis of textures with small scales oscillations, or micro-textures, such as sand or stone.

In this work, we propose to rely on a variational approach to build a synthesis method that allows at the same time for the reconstruction of geometric textures, thanks to the use of sparse, patch-based dictionary, and the reconstruction of small scale oscillations, thanks to constraints on the spectrum of images.
Before presenting in more details the contributions of this paper, and in order to explain both its inspiration and its originality, we start with a short presentation of previous works in this field.


%%%%%%%%%%%%%%%%%%%%%%%%%%%%%%%%%%%%%%%%%%%%%%%%%%%%%%%%%%%%%%%%%%%%%%%%%%%%%%%%
\subsection{Previous Works}
\label{sub:prev-works}


\subsubsection{Patch-Based Approaches}

The first methods yielding realistic texture synthesis rely on a Markovian hypothesis: the distribution of a pixel given the rest of the texture only depends on the values in a (small) neighborhood.
These methods are mostly parametric, following the work in~\cite{cross1983markov}.
The law of a pixel given its neighborhood is chosen \emph{a priori} and its parameters can be learned from an exemplar using a maximum-likelihood estimation.

Non-parametric approaches~\cite{efros1999texture,wei2000fast} were introduced later to handle different kind of textures without designing the law of each of them by hand.
For a given pixel, its value is sampled directly from an exemplar image, by seeking pixels whose square neighborhood (called a {\it patch}) is similar to the one of the pixel to be synthesized.
This approach was a clear breakthrough in the field of texture synthesis, producing results that are visually almost perfect on difficult and structured examples.
It was followed by a large body of works and many extensions were proposed, to perform real-time~\cite{lefebvre2005parallel}, dynamic, or volumetric texture synthesis, for instance.
In particular, it is common to sample a small square area (a patch) at a time, instead of a single pixel~\cite{efros2001image}.
The reader may refer to the state-of-the-art~\cite{Wei2009} for an overview of these approaches applied in different contexts.

One common drawback of these approaches, however, is that the resulting texture is often a plain juxtaposition of small pieces from the exemplar, even when sampling pixels one at a time~\cite{aguerrebere2013exemplar}.
In this sense they have a limited innovation capacity.
An efficient way to avoid this drawback is to perform the synthesis from a dictionary of patches that is learned from the exemplar.
Sparse decomposition methods such as~\cite{elad2006image} are classically applied to image denoising or enhancement.
They rely on the assumption that each patch of an image can be decomposed as a sum of a small number of elements from a dictionary.
The dictionary is usually learned from the image to be restored.
These approaches were introduced in~\cite{olshausen1996natural} as an efficient coding of natural images, inspired by the human vision system.
More efficient algorithms for dictionary learning were proposed later: for instance the Method of Optimal Direction~(MOD)~\cite{engan1999method}, and the K-SVD algorithm~\cite{aharon2006ksvd}.
Recently, one author of the present paper has shown that this framework is well suited to texture modeling~\cite{peyre2009sparse}.
In particular, sparse dictionary learning has the remarkable ability to adaptively extract basic elements, or \emph{textons}, from a texture.
In~\cite{peyre2009sparse}, it is shown that new texture patches can be created from a learned dictionary, simply by imposing a sparsity constraint on their use of the dictionary atoms.


\subsubsection{Statistical Approaches}

Another set of approaches is based on statistical constraints.
A set of statistics is identified and the synthesis process consists in generating a new image whose selected statistics match those of the original texture.
The basic principle of the approach is in agreement with early works on texture discrimination~\cite{Julesz1981}, that spotted statistical constraints as a fruitful tool to investigate the human perception of textures.

In 1995, Heeger and Bergen~\cite{Heeger1995} also proposed to rely on some statistical constraints to perform texture synthesis.
Starting from an exemplar image, a new image is generated by imposing the marginal of wavelet coefficients, separately at each scale and orientation, as well as the image color distribution.
The resulting method, although limited to relatively weakly structured textures, has the ability to produce realistic results in a computationally efficient manner.
Later, it has been proposed~\cite{portilla2000parametric} to impose some second order statistics on wavelet coefficients.
More precisely, the constraints are based on the correlations between coefficients of atoms that are neighbor, either at a given scale or across scales.
This allows for a better preservation of structured textures.
To the best of our knowledge, this relatively old method is the one permitting to synthesize the largest class of textures, without simply copy-pasting pieces from an exemplar.

In a different direction, one may synthesize a texture by imposing constraints on its Fourier transform.
In~\cite{galerne2011random}, it is shown that an efficient synthesis method is achieved by imposing the spectrum of synthesized images, through {\it random phase} textures.
The resulting synthesis is fast and reliable, but limited to non-structured textures with small scales oscillations.
Several works have extended this approach, either simplifying the reference from which the synthesis is performed~\cite{desolneux2012compact,galerne2012gabor} or developing video-texture synthesis methods~\cite{xia2012compact}.

In the field of computer graphics, procedural noises are widely used to yield a realistic aspect for materials.
Contrary to other methods, a procedural noise is generated on-the-fly in any point of $\R^n$, and thus has a very low memory requirement and can be evaluated at any scales with any resolution.
The Perlin noise~\cite{perlin1985image} is the most classical example of such a noise.
It is obtained as a linear combination of colored noises at different scales, resulting in a multi-scale noise whose spectrum is controlled by the weights of the combination.
Sparse convolutive noise is another class of procedural noise, defined as the convolution of a compact kernel with a sparse convolution map.
The Gabor noise is a particular case, using Gabor kernels: the distribution of scales and frequencies of the kernels can be either imposed by hand~\cite{lagae2009procedural} or learned from a sample texture~\cite{galerne2012gabor}.
The state-of-the-art~\cite{lagae2010state} provides a complete review of procedural noise functions.


\subsubsection{Variational Formulation for Texture Synthesis}

In this work, we advocate a generic variational approach to texture synthesis, as presented in detail in Sect.~\ref{sec:variational}.
This is not the first such approach in the context of texture manipulations.
Previous variational approaches aim at transferring the texture of an image onto the content of another image.
The functional to be minimized take into account both the fidelity to the texture image and the fidelity to the content image.
The method presented in~\cite{kwatra2005texture} optimizes an image so that all its patches look similar to a texture given as an example.
The transfer method~\cite{ramanarayanan2007constrained} generates an image $A'$ following the description: ``$A'$ must be to $A$ what $B'$ is to $B$''.
% The image $B'$ is a texture image and the images $A$ and $B$ are content or label images.
% The functional to be minimized measure how close $A'$ is close to $B'$ in the pixel where $A$ is close to $B$.


%%%%%%%%%%%%%%%%%%%%%%%%%%%%%%%%%%%%%%%%%%%%%%%%%%%%%%%%%%%%%%%%%%%%%%%%%%%%%%%%
\subsection{Contributions of the Paper}
\label{sub:approach}

In this paper, we introduce a generic texture synthesis me\-thod that is both suited to highly structured textures and textures that exhibit small scale oscillations.
% The proposed approach also has the ability to truly synthesize new textures, without performing piecewise copy of the exemplar.

First, we rely on adaptive sparse dictionary for the synthesis, following the ideas presented in~\cite{peyre2009sparse}.
As in this this original approach, we impose a sparsity constraint on a decomposition of the patches into a dictionary.
In addition, we also impose that atoms from the dictionary are used in the same proportions as in the original texture samples.
This allows for a faithful reproduction of the structured parts (ed\-ges, corners, etc.) of the input texture.
Second, we impose spectrum constraints on the synthesis, in the manner of~\cite{galerne2011random}.
This allows for the reproduction of high frequency components of the texture, but also, more surprisingly, for the low frequency regularity of the texture, as demonstrated in the experimental section.
We also impose a constraint on the global color content of the image, that has a strong visual impact on the results.
In order to combine the strength of the adaptive patch dictionary to reproduce geometry and the fidelity of frequency reproduction offered by spectrum constraints, we rely on a variational approach.
Each constraint is enforced by controlling the distance of the synthesis to a set of compatible images.
The minimization of the corresponding energy is performed using an averaged projections method.
The approach is compared to the more classical alternating projection method proposed in~\cite{portilla2000parametric}, and the benefit of the proposed constraints is demonstrated.
The resulting synthesis approach achieves, to the best of our knowledge, the best results obtained so far that truly generate a new texture without performing copy-pasting of the input.
Another interesting asset of the proposed approach is that it only relies on first order statistical constraints between atoms from a dictionary.
On the contrary, methods relying on a fixed dictionary necessitates second order statistics to synthesize structured textures~\cite{portilla2000parametric}.
This somehow accredits the second psychophysical theory proposed by B.~Julesz~\cite{Julesz1981}, stating that texture discrimination is related to first order statistics between {\it textons}.
Observe nevertheless that the synthesis also relies on second order statistics between pixel values (through the power spectrum), the\-re\-fore establishing a link between Julesz first~\cite{julesz1962visual} and second~\cite{Julesz1981} theory for texture discrimination.

The plan of the paper is as follows.
We define in Sect.~\ref{sec:variational} the energy function whose minimization leads our synthesis method and present in details the three constraints we impose for the synthesis:  on the histogram, on the spectrum, and on the sparse decomposition of image patches.
We then propose in Sect.~\ref{sec:framework} an algorithm to find a critical point of this energy function.
Numerical results are shown, parameters are discussed and comparisons with the state of the art are provided in Sect.~\ref{sec:results}.
Preliminary results of our work were presented in the conference paper~\cite{tartavel2013constrained}.
More results can be found at: \\
  {\small\url{http://perso.enst.fr/~tartavel/research/jmiv14.html}}, \\
and the Matlab code used for the experiments is publicly available at: \\
  {\small\url{https://bitbucket.org/gtartavel/variational_synthesis}}.

% \rednote[Warning]{These url may change.}


%%%%%%%%%%%%%%%%%%%%%%%%%%%%%%%%%%%%%%%%%%%%%%%%%%%%%%%%%%%%%%%%%%%%%%%%%%%%%%%%
\subsection{Notations}
\label{sub:notations}

A matrix $A=(A_j^i)_{i,j}$ is made of columns $A_j$ and rows $A^i$.
The $\ell^0$ pseudo-norm of a vector $b$ is $\norm{b}_0 = \#\setof{i\suchthat b_i\neq 0}$.

The indicator function $\indic{}$ of a set $\constr{}$ is equal to $0$ on $\constr{}$ and $+\infty$ outside.
The distance of a point $x\in\R^m$ to a compact set $\constr{}$ is $\dist(x,\constr{}) = \min_{y\in\constr{}} \norm{x-y}$ and a projection $\proj_\constr{}(x)$ of $x$ on $\constr{}$ is a minimizer of this distance.

The orthogonal discrete Fourier transform of an image $u$ defined on a rectangular domain $\Omega\subset\Z^2$ of size $M_1\times M_2$ is the image $\hu$ made of the coefficients
\begin{equation}
  \label{eq:dft}
  \hu(m) = \frac{1}{\sqrt{M_1 M_2}} \sum_{x\in\Omega} u(x) \;
    \exp\Big\{ -2\ii\pi \big(\frac{x_1 m_1}{M_1} + \frac{x_2 m_2}{M_2}\big) \Big\}.
\end{equation}

The patches of size $\tau\times\tau$ in $u$ are defined as
\begin{equation}
  p_k = \big( u(x_k+t) \big)_t
    \quad\text{for}\quad
  t\in\setof{0,\dots,\tau-1}^2.
\end{equation}
Their locations $x_k$ lie on a regular grid $x_k\in\Delta\Z^2$ of step $\Delta>0$.
We denote by $\Pi(u) = \big(p_k\big)_k \in\R^{L\times K}$ the matrix, made of the $K$ patches of $u$, where $L=d\tau^2$ is their dimension, with $d=1$ for gray images and $d=3$ for color.
The adjoint operator of $\Pi$ is denoted by $\Pi^*$.
% We define $Z=\lceil\frac{\tau}{\Delta}\rceil^2$ which satisfies $\norm{\Pi}^2_{2,2}\leq Z$.