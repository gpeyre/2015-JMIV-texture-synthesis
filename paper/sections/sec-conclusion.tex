\section{Conclusion}
\label{sec:conclusion}

In this paper, we proposed a texture synthesis framework which puts together both a Fourier spectrum constraint and a sparsity constraint on the patches.
Thanks to these two constraints, our method is able to synthesize a wide range of textures without a patch-based copy-paste approach.
The spectrum controls the amount of oscillations and the grain of the synthesized image,
while the sparsity of the patches handles geometrical information such as edges and sharp patterns.

We propose a variational framework to take both constraints into account.
This framework is based on a non-convex energy function defined as the sum of weighted distances between the image (or some linear transform of it) and these constraints.
The synthesis consists in finding local minima of this function.
The proposed framework is generic and can take into account other constraints: in addition to the sparsity and spectrum constraint, we use a histogram constraint to control the color distribution.